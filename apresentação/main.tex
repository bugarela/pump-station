\documentclass{beamer}

\usepackage{graphicx,hyperref,udesc,url}
\usepackage{amsmath,amsthm,amsfonts,amssymb}
\usepackage{latexsym}
\usepackage{graphicx,udesc,url}
\usepackage[utf8]{inputenc}
\usepackage[T1]{fontenc}
\usepackage{booktabs}
\usepackage{gensymb}
\usepackage{amsmath,amssymb,proof}
\usepackage{booktabs}
\usepackage{dirtree}
\usepackage{listings}
\usepackage{lmodern}
\usepackage{natbib,bibentry}

\definecolor{green-hl}{RGB}{54,88,65}
\newcommand{\hl}[1]{{\color{white}\colorbox{green-hl}{#1}}}

\input{meta.tex}
\input{tla.tex}

% \lstset{
%     numbers=left,                   % where to put the line-numbers
%     numberstyle=\small \ttfamily \color[rgb]{0.4,0.4,0.4},
%                 % style used for the linenumbers
%     showspaces=false,               % show spaces adding special underscores
%     showstringspaces=false,         % underline spaces within strings
%     showtabs=false,                 % show tabs within strings adding particular underscores
%     frame=lines,                    % add a frame around the code
%     tabsize=4,                        % default tabsize: 4 spaces
%     breaklines=false,                % automatic line breaking
%     breakatwhitespace=false,        % automatic breaks should only happen at whitespace
%     basicstyle=\ttfamily,
%     %identifierstyle=\color[rgb]{0.3,0.133,0.133},   % colors in variables and function names, if desired.
%     keywordstyle=\color[rgb]{0.133,0.133,0.6},
%     commentstyle=\color[rgb]{0.133,0.545,0.133},
%     stringstyle=\color[rgb]{0.627,0.126,0.941},
% }

\sloppy

\title[Controlador de sistema de
  bombeamento de água]{Epecificação formal e prototipação de controlador de sistema de bombeamento de água}
\author[Gabriela M. Mafra]{
    Gabriela Moreira Mafra\\\smallskip
    {\scriptsize Universidade do Estado de Santa Catarina \\\smallskip
    \vspace{-2mm}
    \url{gabrielamoreiramafra@gmail.com}}
}

\begin{document}
  \date{23 de Julho de 2020}
  \begin{frame}
      \titlepage
  \end{frame}

  \section{Contexto}

  \begin{frame}
    \frametitle{Estação de bombeamento de água}

      \begin{itemize}
      \item Várias bombas de água abastecem uma reserva
      \item Bombas podem ser ativadas e desativadas com um custo associado.
      \item Nível da água da reserva deve se manter nos limites estabelecidos.
      \item As bombas tem particularidades.
      \end{itemize}
  \end{frame}

  \begin{frame}
    \frametitle{Algoritmo proposto}
      Para obter o algoritmo, é feita uma
simulação do ambiente, onde o fluxo de água é simulado a partir da capacidade
das bombas, e um algoritmo genético faz a seleção do controlador ótimo.

     \begin{figure}
        \includegraphics[width=0.6\textwidth]{obtencao_algoritmo.png}
        \caption{Método de obtenção do algoritmo. Fonte: \cite{pumps}}
        \label{fig:ams}
      \end{figure}

  \end{frame}

  \begin{frame}
    \frametitle{Algoritmo proposto - ANSI C}
    \begin{minipage}{0.45\textwidth}
      \begin{figure}
        \includegraphics[width=\textwidth]{select_pumps_ansi.png}
        \caption{Algoritmo de seleção de bombas. Fonte: \cite{pumps}}
        \label{fig:ams}
      \end{figure}
    \end{minipage}
    \hfill
    \begin{minipage}{0.5\textwidth}\raggedleft
      \begin{figure}
        \includegraphics[width=\textwidth]{switch_pumps_ansi.png}
        \caption{Algoritmo de mudança de bombas. Fonte: \cite{pumps}}
        \label{fig:ams}

      \end{figure}
    \end{minipage}

  \end{frame}

  \section{Proposta}
  \begin{frame}
    \frametitle{Objetivos}

    \begin{itemize}
    \item Especificar o algoritmo de controle proposto em \cite{pumps} usando a liguagem \TLA;
    \item Gerar código executável em Elixir a partir dessa especificação, usando a
      ferramenta proposta em \cite{tcc};
    \item Adicionar módulo de comunicação com protocolo MQTT para receber dados de
      sensores;
    \item Simular o envio de dados de sensores, verificando se as reações do
      controlador estão de acordo com o esperado para o algoritmo.
    \end{itemize}
  \end{frame}

  
  \begin{frame}
    \frametitle{Benefícios da Espeficificação em \TLA}

    \begin{minipage}{0.45\textwidth}
      \begin{figure}
  \includegraphics[width=\textwidth]{activate_ansi.png}
  \caption{Especificação da priorização de bombas a serem ativadas em ANSI C.
    Fonte: \cite{pumps}}
      \end{figure}

    \end{minipage}
    \hfill
    \begin{minipage}{0.5\textwidth}\raggedleft
      \begin{itemize}
      \item As bombas 0, 1 e 2 devem ter uso revezado
      \item As bombas 3 e 4 só devem ser usadas em emergências
      \item Controlado por contadores
      \end{itemize}
    \end{minipage}
    \end{frame}
    
    \begin{frame}
    \frametitle{Benefícios da Espeficificação em \TLA (Continuação)}

      \begin{figure}
  \includegraphics[width=\textwidth]{activate_tla.png}
  \caption{Especificação da priorização de bombas a serem ativadas em \TLA.
    Fonte: autora}
      \end{figure}


  \end{frame}
  
    \begin{frame}
    \frametitle{Geração de código e Comunicação}

      \begin{figure}
  \includegraphics[width=\textwidth]{diagrama.png}
  \caption{Processo de geração de código e adição do módulo de comunicação.
    Fonte: autora}
      \end{figure}


  \end{frame}

\section{Resultados}
  
  \begin{frame}
    \frametitle{Resultado}

    Demonstração
    \begin{itemize}
    \item Raspberry Pi 2B V1.1 (Controlador + Broker)
    \item Sensores simulados com $\texttt{mosquitto\_pub}$ de outro computador
      \end{itemize}
      \end{frame}
  
  \begin{frame}
    \frametitle{Considerações Finais}

    Usar métodos formais é uma forma mais adequada de especificar algoritmos
    \begin{itemize}
    \item Explicita propriedades
    \item Possibilida verificações
    \end{itemize}
    Ainda mais vantajoso quando se pode obter um protótipo.\bigskip\\\pause

    \textbf{Trabalhos Futuros}
    \begin{itemize}
    \item Verificar propriedades complexas
    \item Propor essa abordagem de especificação para mais sistemas,
      principalmente distribuídos.
    \end{itemize}
  \end{frame}
  

  \begin{frame}
    \frametitle{Referências}
    
    \bibliographystyle{abnt-alf}
    \bibliography{main}
    
  \end{frame}

  \begin{frame}
    \frametitle{Obrigada!}
    
    Gabriela Moreira Mafra\\
    Universidade do Estado de Santa Catarina \\\medskip
    \vspace{-2mm}
    \url{gabrielamoreiramafra@gmail.com}



  \end{frame}

\end{document}

